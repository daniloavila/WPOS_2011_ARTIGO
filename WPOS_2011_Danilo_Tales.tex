\documentclass[conference]{IEEEtran}
\usepackage{makeidx}  % allows for indexgeneration
\usepackage[english,brazil,brazilian,portuguese]{babel} %% Usado na qualifica��o
%%%%%%%%%%%%%%%%%%%%%%%%%%%%%%%%%%%%%%
% O pacote na linha abaixo n�o hifena corretamente palavras acentuadas
%\usepackage[latin1]{inputenc}

% correct bad hyphenation here
\hyphenation{op-tical net-works semi-conduc-tor}
\hyphenation{register-Listener}
\hyphenation{he-te-ro-gei-ni-da-de}
\hyphenation{co-necti-vi-da-de}
\hyphenation{en-capsu-lan-do}
\hyphenation{auxi-liam}
\hyphenation{mo-bi-li-da-de}
\hyphenation{cons-tru-��o}
\hyphenation{ex-pe-ri�n-cia}
\hyphenation{res-pon-s�-vel}
\hyphenation{ca-mi-nho}
\hyphenation{middleware}
%
\usepackage[T1]{fontenc}%%Hifena corretamente palavras acentuadas
\usepackage{ae} %% pra melhorar a gera��o de pdf do comando acima
\usepackage[numbers]{natbib}
\usepackage{url} %% Para renderizar Urls
\usepackage{graphicx} %% Para utilizar imagens jpeg
%\usepackage{hyperref} %% para utilizar casos especiais de ref, como o \autoref{}, al�m de criar links especiais para os itens do documento
\usepackage{rotating} %% Para rotacionar elementos
\usepackage{listings} %% Para exibir trechos de c�digo

\begin{document}
%
% paper title
% can use linebreaks \\ within to get better formatting as desired
\title{Identifica��o e Localiza��o de Pessoas em ambientes inteligentes.}


% author names and affiliations
% use a multiple column layout for up to three different
% affiliations
\author{\IEEEauthorblockN{Danilo �vila Monte Christo Ferreira, Tales Mundim de Andrade Porto}
\IEEEauthorblockA{Departamento de Ci�ncias da Computa��o , Instituto de Ci�ncias Exatas\\
Universidade de Bras�lia (UnB)\\
Bras�lia, Brazil\\
Campus Universit�rio Darcy Ribeiro - Asa Norte -- ICC Centro -- Caixa postal 4466\\
70.910-900 -- Bras�lia -- DF -- Brasil\\
daniloavilaf@gmail.com, talesap@gmail.com\\ 
\texttt{http://www.cic.unb.br}}
}


% make the title area
\maketitle


\begin{abstract}
Abstract
\end{abstract}

\section{Introdu��o}
\label{sec:introduction}
Introdu��o

\section{Sistema TRUE}
\label{sec:proposta}

Proposta

\section{Testes}
\label{sec:testes}

Testes

\section{Conclus�o}
\label{sec:conclusao}

Conclus�o

\section{Desafios futuros}
\label{sec:desafios}

Desafios Futuros


\bibliographystyle{IEEEtran}
\bibliography{IEEEabrv,WPOS_2011_Danilo_Tales}


% that's all folks
\end{document}
